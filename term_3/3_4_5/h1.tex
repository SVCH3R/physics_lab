\documentclass[a4paper,12pt]{article}
\usepackage{style}


\title{
    Лабораторная работа 3.4.5 Петля гистерезиса (динамический метод) \\
    Теория Алгоритмов и Моделей Вычислений.\\}
\author{Севастьян Черняков Б05-207}
\date{2023-2024}


\begin{document}

\maketitle



\vspace{15pt}

\hrule 
\vspace{15pt}

\section{Введение}

\textbf{Цель работы:} 

	Изучение петель гистерезиса различных ферромагнитных
материалов в переменных полях.


\textbf{В работе используются:} 

Автотрансформатор, понижающий трансформатор, интегрирующая цепочка, 

амперметр, вольтметр, 

электронный
осциллограф, делитель напряжения, тороидальные образцы с двумя обмотками.


\section{Результаты измерений и обработка данных}

\subsection{Исследование петли гистерезиса}
Параметры установки следующие: $R_0 = 0,3Ом$, $R_и = 20 \; кОм$, $C_и = 20 \; мкФ$, $\omega = 50\; Гц$. 
Подберем ток питания в намагничивающей обмотке с помощью автотрансформатора и коэффициенты усиления ЭО
таким образом, чтобы предельная петля гистерезиса занимала большую часть экрана. Приведем характерные 
значения катушек разных материалов в таблице.



\vspace{15pt}


\end{document}
